\pdfoutput=1
\pdfcompresslevel=9
\pdfinfo
{
    /Author (Autor)
    /Title (Tytul)
    /Subject (Tematyka)
    /Keywords (Slowa kluczowe)
}
%\documentclass[a4paper,polish,onecolumn,oneside,floatssmall,11pt,titleauthor,wide,openright]{mwrep}
%\usepackage[scale={0.7,0.8},paper=a4paper,twoside]{geometry}
\documentclass[a4paper,onecolumn,oneside,11pt,wide,floatssmall]{mwrep}
% \usepackage{polish}
\usepackage{amsmath}
\usepackage{amsfonts}
\usepackage{amssymb}
\usepackage{amsthm}
\usepackage{bookman}

\usepackage{geometry}
\usepackage[utf8x]{inputenc}
\usepackage[T1]{fontenc}
% \usepackage{t1enc}
% \usepackage[pdftex, bookmarks]{hyperref}
\usepackage[pdftex, bookmarks=false]{hyperref}
\def\url#1{{ \tt #1}}

\usepackage{listings}

% marginesy
\textwidth\paperwidth
\advance\textwidth -55mm
\oddsidemargin-0.9in
\advance\oddsidemargin 33mm
\evensidemargin-0.9in
\advance\evensidemargin 33mm
\topmargin -1in
\advance\topmargin 25mm
\setlength\textheight{48\baselineskip}
\addtolength\textheight{\topskip}
\marginparwidth15mm

\clubpenalty=10000 % to kara za sierotki
\widowpenalty=10000 % nie pozostawia wdów
\brokenpenalty=10000 % nie dzieli wyrazów pomiędzy stronami
\sloppy

\tolerance4500
\pretolerance250
\hfuzz=1.5pt
\hbadness1450

% ŻYWA PAGINA
\renewcommand{\chaptermark}[1]{\markboth{\scshape\small\bfseries \
#1}{\small\bfseries \ #1}}
\renewcommand{\sectionmark}[1]{\markboth{\scshape\small\bfseries\thesection.\
#1}{\small\bfseries\thesection.\ #1}}
\newcommand{\headrulewidth}{0.5pt}
\newcommand{\footrulewidth}{0.pt}
\pagestyle{uheadings}

\usepackage[pdftex]{color,graphicx}
\usepackage[polish]{babel}

% \textheight232mm
% \setlength{\textwidth}{\textwidth}
% \setlength{\oddsidemargin}{\evensidemargin}
% \setlength{\evensidemargin}{0.3cm}
\usepackage[sort, compress]{cite}

%\usepackage{multibib}
%\newcites{bk,st,doc,web}{Książki i~artykuły,Standardy i~zalecenia,Dokumentacja produktów,Publikacje i~serwisy internetowe}

\theoremstyle{definition}
\newtheorem{defn}{Definicja}[section]
\newtheorem{conj}{Teza}[section]
\newtheorem{conjmain}{Teza}
\newtheorem{exmp}{Przykład}[section]

\theoremstyle{plain}% default
\newtheorem{thm}{Twierdzenie}[section]
\newtheorem{lem}[thm]{Lemat}
\newtheorem{prop}[thm]{Hipoteza}
\newtheorem*{cor}{Wniosek}

\theoremstyle{remark}
\newtheorem*{rem}{Uwaga}
\newtheorem*{note}{Uwaga}
\newtheorem{case}{Przypadek}

\definecolor{ListingBackground}{rgb}{0.95,0.95,0.95}

\begin{document}

% kody źródłowe wplatane w tekst
\lstdefinestyle{incode}
{
basicstyle={\footnotesize},
keywordstyle={\bf\footnotesize\color{blue}},
commentstyle={\em\footnotesize\color{magenta}},
numbers=left,
stepnumber=5,
firstnumber=1,
numberfirstline=true,
numberblanklines=true,
numberstyle={\sf\tiny},
numbersep=10pt,
tabsize=2,
xleftmargin=17pt,
framexleftmargin=3pt,
framexbottommargin=2pt,
framextopmargin=2pt,
framexrightmargin=0pt,
showstringspaces=true,
backgroundcolor={\color{ListingBackground}},
extendedchars=true,
% title=\lstname,
captionpos=b,
% abovecaptionskip=1pt,
% belowcaptionskip=1pt,
frame=tb,
framerule=0pt,
}

% kody źródłowe z podpisem
\lstdefinestyle{outcode}
{
basicstyle={\footnotesize},
keywordstyle={\bf\footnotesize\color{blue}},
commentstyle={\em\footnotesize\color{magenta}},
numbers=left,
stepnumber=5,
firstnumber=1,
numberfirstline=true,
numberblanklines=true,
numberstyle={\sf\tiny},
numbersep=10pt,
tabsize=2,
xleftmargin=17pt,
framexleftmargin=3pt,
framexbottommargin=2pt,
framextopmargin=2pt,
framexrightmargin=0pt,
showstringspaces=true,
backgroundcolor={\color{ListingBackground}},
extendedchars=true,
% title=\lstname,
captionpos=b,
% abovecaptionskip=1pt,
% belowcaptionskip=1pt,
frame=tb,
framerule=0.1pt,
}

\renewcommand*\lstlistingname{Wydruk}
\renewcommand*\lstlistlistingname{Spis wydruków}

\pagenumbering{roman}
\renewcommand{\baselinestretch}{1.0}
\raggedbottom

\begin{titlepage}
    % Strona tytułowa
    \vbox to\textheight{\hyphenpenalty=10000
    \begin{center}
	\begin{tabular}{p{107mm} p{9cm}}
	    \begin{minipage}{9cm}
	      \begin{center}
	      Politechnika Warszawska \\
	      Wydział Elektroniki i~Technik Informacyjnych \\
	      Instytut Informatyki
	      \end{center}
	    \end{minipage}
	    &
	    \begin{minipage}{8cm}
	    \begin{flushleft}
	     \footnotesize
	      Rok akademicki 200X/200Y
	    \vspace*{2.75\baselineskip}
	    \end{flushleft}
	    \end{minipage} \\
	\end{tabular}
	\vspace*{3.75\baselineskip}
	\par\vspace{\smallskipamount}
	\vspace*{2\baselineskip}{\LARGE Praca dyplomowa magisterska\par}
	\vspace{3\baselineskip}{\LARGE\strut Imię i Nazwisko\par}
	\vspace*{2\baselineskip}{\huge\bfseries Tytuł pracy\par}

	\vspace*{7\baselineskip}
	\hfill\mbox{}\par\vspace*{\baselineskip}\noindent
	\begin{tabular}[b]{@{}p{3cm}@{\ }l@{}}
	    {\large\hfill } & {\large }
	\end{tabular}
	\hfill
	\begin{tabular}[b]{@{}l@{}}
	Opiekun pracy: \\[\smallskipamount]
	{\large Tytuł Imię i Nazwisko}
	\end{tabular}\par
	\vspace*{4\baselineskip}
    \begin{tabular}{p{\textwidth}}
    \begin{flushleft}
	\begin{minipage}{7cm}
	Ocena \dotfill
	\par\vspace{1.6\baselineskip}
	\dotfill
	\par\noindent
	\centerline{\footnotesize Podpis Przewodniczącego} \par
	\centerline{\footnotesize Komisji Egzaminu Dyplomowego}\par
	\end{minipage}
    \end{flushleft}
    \end{tabular}
    \end{center}}

    % Życiorys
    \newpage\thispagestyle{empty}
    \begin{tabular}{p{5cm} p{12cm}}
    \begin{minipage}{5cm}
    \center
    \includegraphics[height=6.5cm,width=4.5cm]{img/foto.jpg}
    \end{minipage}
    &
    \begin{minipage}{12cm}
    \begin{flushleft}
    \par\noindent\vspace{1\baselineskip}
    \begin{tabular}[h]{l l}
    {\normalsize\it Specjalność:} & Informatyka -- \\
    & Inżynieria oprogramowania \\
    & i~systemy informacyjne
    \end{tabular}
    \par\noindent\vspace{1\baselineskip}
    \begin{tabular}[h]{l l}
    {\normalsize\it Data urodzenia:} & {\normalsize 1 stycznia 1980~r.}
    \end{tabular}
    \par\noindent\vspace{1\baselineskip}
    \begin{tabular}[h]{l l}
    {\normalsize\it Data rozpoczęcia studiów:} & {\normalsize 1 października 2002 r.}
    \end{tabular}
    \par\noindent\vspace{1\baselineskip}
    \end{flushleft}
    \end{minipage}
    \end{tabular}
    \vspace*{1\baselineskip}
    \begin{center}
	{\large\bfseries Życiorys}\par\bigskip
    \end{center}

    \indent
    Nazywam się  \ldots.
    \par
    \vspace{2\baselineskip}
    \hfill\parbox{15em}{{\small\dotfill}\\[-.3ex]
    \centerline{\footnotesize podpis studenta}}\par
    \vspace{3\baselineskip}
    \begin{center}
 	{\large\bfseries Egzamin dyplomowy} \par\bigskip\bigskip
    \end{center}
    \par\noindent\vspace{1.5\baselineskip}
    Złożył egzamin dyplomowy w dn. \dotfill
    \par\noindent\vspace{1.5\baselineskip}
    Z wynikiem \dotfill
    \par\noindent\vspace{1.5\baselineskip}
    Ogólny wynik studiów \dotfill
    \par\noindent\vspace{1.5\baselineskip}
    Dodatkowe wnioski i uwagi Komisji \dotfill
    \par\noindent\vspace{1.5\baselineskip}
    \dotfill

    % Streszczenie
    \newpage\thispagestyle{empty}
    \vspace*{2\baselineskip}
    \begin{center}
	{\large\bfseries Streszczenie}\par\bigskip
    \end{center}

    {\itshape
    Praca ta prezentuje \ldots}
    \vspace*{1\baselineskip}

    \noindent{\bf Słowa kluczowe}: {\itshape słowa kluczowe.}
    \par
    \vspace{4\baselineskip}
    \begin{center}
	{\large\bfseries Abstract}\par\bigskip
    \end{center}
    \noindent{\bf Title}: {\itshape Thesis title.}\par
    \vspace*{1\baselineskip}
    {\itshape
    This thesis describes \ldots}
    \vspace*{1\baselineskip}

    \noindent{\bf Key words}: {\itshape key words.}

\end{titlepage}
\end{document}
% ex: set tabstop=4 shiftwidth=4 softtabstop=4 noexpandtab fileformat=unix filetype=tex spelllang=pl,en spell:
