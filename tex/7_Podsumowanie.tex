\chapter{Podsumowanie}
W ramach niniejszej pracy zrealizowany został prototyp aplikacji służącej do obsługi umów cywilno-prawnych. Pozwala on nie tylko na obsługę umów, ale również pracowników na nie zatrudnianych czy zadań w ramach, których są podpisywane. Umożliwia również korzystanie z rozbudowanego modelu organizacji instytucji oraz zapewnia podstawowy mechanizm bezpieczeństwa. Aplikacja oparta została o model trójwarstwowy. Cel pracy został więc osiągnięty.

Efekt ten udało się uzyskać dzięki przejściu przez wszystkie etapy wytwarzania oprogramowania. Pierwszym z nich było zapoznanie się z dziedziną problemu oraz zapoznanie się z dostępnymi na rynku rozwiązaniami. Zostało to opisane w rozdziale \ref{chap2}. Następnie należało przeanalizować wymagania stawiane przed aplikacją, zidentyfikować aktorów oraz wyodrębnić przypadki użycia, co zostało udokumentowane w rozdziale \ref{chap3}. Kolejnym etapem było zapoznanie się z listą narzędzi służących do tworzenia aplikacji w architekturze trójwarstwowej, opisanych w rozdziale \ref{chap4}. Następnie należało przystąpić do najważniejszego etapu pracy nad system jakim jest opracowanie modelu danych. W efekcie powstały diagramy związków encji oraz schematy tabel zaprezentowane w rozdziale \ref{chap5}. Zwieńczeniem całego procesu wytwarzania oprogramowania jest etap implementacji. Został on opisany w rozdziale \ref{chap6}.
	
Dzięki zastosowaniu sprawdzonych narzędzi oraz popularnej architektury trójwarstwowej dalszy rozwój aplikacji jest nie tylko możliwy, ale nie wymaga też dużych nakładów pracy. Zważywszy, że dziedzina problemu jest bardzo szeroka, aplikacja może być w przyszłości wzbogacana o kolejne funkcjonalności pozwalające na jeszcze lepszą obsługę umów cywilno-prawnych.
