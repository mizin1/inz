\chapter{Podsumowanie}
Celem niniejszej pracy było zaprojektowanie oraz implementacja prototypu trójwarstwowej aplikacji internetowej wspomagającej obsługę umów cywilno-prawnych. Pierwszym jej etapem było zapoznanie się z dziedziną zagadnienia. Poskutkowało to stworzeniem modelu funkcjonalnego. Dokonano również przeglądu dostępnych na rynku rozwiązań. Następnie zaprojektowano model danych, który stanowił szkielet realizowanego systemu. Po zapoznaniu się z dostępnymi narzędziami, przystąpiono do fazy implementacji. 

Rezultatem tych działań jest zrealizowana aplikacja. Spełnia ona wymagania zarówno uczelni jak i innych instytucji. Jej podstawowym zadaniem jest obsługa umów cywilno-prawnych. Zapewnia też wiele dodatkowych funkcjonalności, takich jak zarządzanie pracownikami czy zadaniami wykonywanymi w ramach umów. Ważną jej cechą jest rozbudowany model organizacji instytucji oraz ściśle z nim zintegrowany mechanizm bezpieczeństwa. Podsumowując, cel pracy został osiągnięty.

\section[Możliwości rozwoju][Możliwości rozwoju]{Możliwości rozwoju}
System zaprojektowany został z myślą o jego dalszym rozwoju. Zastosowanie elastycznego modelu danych pozwala na łatwe dodawanie nowych typów umów. Możliwe jest rozszerzenie funkcjonalności aplikacji o obsługę różnego rodzaju załączników i aneksów. Ciekawym pomysłem wydaję się też wprowadzenie deklaracji PIT czy ZUS. 



%Zastosowanie popularnej architektury trójwarstwowej oraz elastycznego modelu danych ułatwia dalszy rozwój aplikacji. 
%Pozwala ona przede wszystkim na obsługę umów cywilno-prawnych. Zapewnia też wiele dodatkowych funkcjonalności. Pozwala na zarządzanie zatrudnianymi na umowy pracownikami oraz zadaniami, na których wykonywanie podpisywane są umowy.. Spełniają one potrzeby zarówno uczelni jaki i innych instytucji tego typu. Ze względu na internetowy charakter, dostęp do aplikacji jest możliwy w dowolnym miejscu i czasie. . Cel pracy został więc osiągnięty.
