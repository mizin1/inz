\chapter{Wprowadzenie}

\section[Motywacja][Motywacja]{Motywacja}
W okresie kryzysu gospodarczego coraz popularniejsze stają się alternatywne formy zatrudnienia jakimi są jak umowy cywilno-prawne. Służą one nie tylko do redukowania kosztów zatrudnienia. Mają zastosowanie wszędzie tam, gdzie pracodawcy chcą uniknąć ograniczeń będących następstwem podpisania umowy o pracę. Wykorzystywane są również w sytuacjach, w których zachodzi potrzeba powierzenia dodatkowych zadań pracownikom już zatrudnionym na etat.

Umowy cywilno-prawne wykorzystuje także Politechnika Warszawska. Posługuje się nimi Ośrodek Kształcenia na Odległość, w którym to pojawiła się potrzeba automatyzacji procesu ich obsługi. Mimo że na rynku są już obecne narzędzia do tego przeznaczone, zdecydowano się na stworzenie dedykowanego rozwiązania dostosowanego do specyfiki uczelni.

W wyniku rozwoju sieci Internet, coraz powszechniejsze stają się aplikacje webowe. Do ich największych zalet należą niezależność od systemu operacyjnego oraz łatwość aktualizacji. Ze względu na rosnącą popularność takich rozwiązań zdecydowano się umieścić projektowany system w gronie tego typu aplikacji.

\section[Cel pracy][Cel pracy]{Cel pracy}
Celem pracy jest zaprojektowanie oraz implementacja prototypu trójwarstwowej aplikacji internetowej wspomagającej obsługę umów cywilno-prawnych, przeznaczonej dla uczelni. Aplikacja ta powinna oferować funkcjonalności związane z obsługą umów i zatrudnianych na nie pracowników oraz uwzględniać strukturę organizacji, dla której jest przeznaczona.



%Potrzeba  takich umów pojawiła się w Ośrodku Kształcenia na Odległość Politechniki Warszawskiej. Mimo iż na rynku są już obecne narzędzia do tego przeznaczone, zdecydowano się na stworzenie dedykowanego rozwiązania dostosowanego do specyfiki uczelni.
