\chapter{Wprowadzenie}

\section[Motywacja][Motywacja]{Motywacja}
W okresie kryzysu gospodarczego pracy coraz częstsze stają się alternatywne formy zatrudnienia takie jak umowy cywilno-prawne. Służą one nie tylko do redukowania kosztów pracy, ale mają zastosowanie wszędzie tam gdzie pracodawcy nie chcą nawiązywać trwałego stosunku pracy na podstawie bezterminowej umowy o pracę. Doskonale sprawdzają się w przypadku zlecania małych zadań wykonywanych również dla własnego pracodawcy. Pojawiają się także na Politechnice Warszawskiej. Wraz z narastająca ich liczbą zaistniała potrzeba automatyzacji procesu obsługi takich umów.

\section[Tytuł w paginie][Tytuł w spisie treści]{Cel pracy}
Celem pracy jest zaprojektowanie oraz implementacja prototypu aplikacji wspomagającej korzystanie z umów cywilno-prawnych, przeznaczonej dla instytucji typu uczelnia. Aplikacja powinna oferować funkcjonalności związane z obsługą umów i zatrudnianych na nie pracowników oraz uwzględniać strukturę organizacji dla której jest przeznaczona. Kolejnym założeniem projektowym jest wykorzystanie architektury trójwarstwowej, tzn podziału aplikacji na warstwę prezentacji(realizowanej przez przeglądarkę internetową), warstwę aplikacji i warstwę źródła danych.
