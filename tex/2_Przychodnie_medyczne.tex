\chapter{Przychodnie medyczne}
Dokupując analizy dziedziny przychodni medycznych nieodłącznym elementem jest konieczność zapoznania się z treściami licznych ustaw. W służbie zdrowia wszystkie ustroje zdefiniowane są przez ustawę. To również ustawodawca określa zakres kompetencji oraz odpowiedzialność każdego z uczestników systemu. Niestety treści ustaw ulegają częstym zmianom co rodzi wiele problemów administracyjnych (przykładem mogą być warunki ubezpieczeń świadczeniodawców). Aby rozpocząć omawianie tematu przychodni medycznych warto określić ich miejsce w strukturze organizacyjnej.

\section[Miejsce w strukturze systemu służby zdrowia][Miejsce w strukturze systemu służby zdrowia]{Miejsce w strukturze systemu służby zdrowia}
System służby zdrowia jest to zbiór ludzi, instytucji i zasobów, których celem jest świadczenie usług zdrowotnych dla wyznaczonych populacji.
Można w nim wyróżnić następujące grupy:
\begin{itemize}
\item świadczeniodawców (podmioty lecznicze, apteki)
\item świadczeniobiorców
\item płatników (np. NFZ)
\item instytucje nadzorujące i kierujące systemem ochrony zdrowia
\end{itemize}

Interesującą dla przychodni medycznych jest po grupa świadczeniodawców - podmioty lecznicze. Podmioty lecznicze w myśl ustawy {mz:udl:online} dzielą się na:
\begin{itemize}
\item przedsiębiorcy 
\item samodzielne publiczne zakłady opieki zdrowotnej
\item jednostki budżetowe
\item instytuty badawcze
\item fundacje i stowarzyszenia
\item kościoły
\end{itemize} 
Dokonując dalszego podziału przedsiębiorców zgodnie z załącznikiem nr 1 do ustawy otrzymujemy następujący podział ze względu na wykonywaną działalność:
\begin{itemize}
\item 1 - Stacjonarne i całodobowe świadczenia szpitalne
\item 2 - Stacjonarne i całodobowe świadczenia zdrowotne inne niż szpitale
\item 3 - Ambulatoryjne świadczenia zdrowotne
\end{itemize}

Zgodnie z treścią ustawy: Ambulatoryjne świadczenia zdrowotne obejmują świadczenia podstawowej lub specjalistycznej opieki zdrowotnej oraz świadczenia z zakresu rehabilitacji leczniczej, udzielane w warunkach niewymagających ich udzielania w trybie stacjonarnym i całodobowym w odpowiednio urządzonym, stałym pomieszczeniu. Udzielanie tych świadczeń może odbywać się w pomieszczeniach przedsiębiorstwa, w tym w pojeździe przeznaczonym do udzielania tych świadczeń, lub w miejscu pobytu pacjenta.
Ambulatoryjnych świadczeń zdrowotnych udziela się w ambulatorium (przychodni, poradni, ośrodku zdrowia, lecznicy lub ambulatorium z izbą chorych), a także zakładzie badań diagnostycznych i medycznym laboratorium diagnostycznym.  
Przychodnie medyczne wchodzą więc w skład podmiotów leczniczych wykonujących ambulatoryjne świadczenia zdrowotne. 

Wejściu w życie ustawy o działalności leczniczej \cite{mz:udl:online} w roku 2012, doprowadziło do licznych zmian administracyjnych w prywatnej części służby zdrowia. Przykładem były placówki zabiegowe jednego dnia, które po wejście w życie ustawy stały się w myśl ustawodawcy szpitalami. Konsekwencją takiej sytuacji były zmiany w umowach z ubezpieczycielami podmiotów leczniczych. Żądali oni od placówek leczniczych parokrotnie wyższych składek ubezpieczeniowych ze względu na prowadzenie działalności szpitala. Spowodowało to liczne zmiany statusów na podmioty lecznicze świadczące ambulatoryjne świadczenia zdrowotne. 


\section[System resortowych kodów identyfikacyjnych][System resortowych kodów identyfikacyjnych]{System resortowych kodów identyfikacyjnych}

\subsection{Organizacja podmiotów leczniczych} 
\label{subsec:org_pod_lecz}
Zanim przedstawiona zostanie struktura resortowych kodów identyfikacyjnych, należy przedstawić jak wygląda ustawowo nadana organizacja podmiotów leczniczych. Przykładem podmiotu może być osoba fizyczna prowadząca działalność medyczną czy też spółka z ograniczoną odpowiedzialnością. Zgodnie z komunikatem w sprawie dostosowania działalności podmiotów leczniczych do przepisów ustawy o działalności leczniczej \cite{mz:kdp:online}, podmiot leczniczy wykonuje działalność leczniczą w przedsiębiorstwie. Przedsiębiorstwo jest to zespół składników majątkowych, poprzez które podmiot podmiot leczniczy wykonuje działalność leczniczą. Jednocześnie w przedsiębiorstwie może być prowadzony jedynie jeden rodzaj działalności leczniczej. Przykładem przedsiębiorstwa może być szpital lub przychodnia. W przedsiębiorstwie wyodrębnione są jednostki lub komórki organizacyjne zgodnie z rozporządzeniem ministra w sprawie szczegółowego zakresu danych objętych wpisem do rejestru podmiotów wykonujących działalność leczniczą oraz szczegółowego trybu postępowania w sprawach dokonywania wpisów, zmian w rejestrze oraz wykreśleń z tego rejestru \cite{mz:zdr:online}. Przykładem komórki organizacyjnej może być oddział szpitalny lub pracownia USG. 

\subsection{Kody resortowe}
Każdy podmiot wykonujący działalność leczniczą posiada nadany resortowy kod identyfikacyjny, którego forma określona jest w rozporządzeniu ministra zdrowia w sprawie systemu resortowych kodów identyfikacyjnych \cite{mz:rki:online}. Taki kod nadawany jest na podstawie informacji zawartych we wniosku o wpis do rejestru. W jego skład wchodzi dziesięć części.

\begin{table}
\centering
\caption{Resortowy kod identyfikacyjny}
\label{tab:card}
\begin{minipage}{.9\textwidth}
\setlength{\baselineskip}{2mm}
\centering
\begin{tabular}{c|c|c}
Część & L. znaków & Opis \\ \hline
{\em I}   & 12 & Numer księgi rejestrowej podmiotu. \\ \hline
{\em II}  & 7  & Identyfikator jednostki podziału terytorialnego, \\
		  &    & w którym znajduje się siedziba podmiotu. \\ \hline
{\em III} & 2  & Kod podmiotu tworzącego, w przypadku podmiotu \\ 
		  &    & leczniczego niebędącego przedsiębiorcą. \\ \hline
{\em IV}  & 4  & Kod określający formę organizacyjno prawną \\
		  &    & podmiotu. \\ \hline
{\em V}   & 2  & Kod identyfikujący jednostkę organizacyjną \\
		  &    & przedsiębiorstwa podmiotu leczniczego. \\ \hline
{\em VI}  & 1  & Kod rodzaju działalności leczniczej wykonywanej \\
		  &    & w przedsiębiorstwie podmiotu. \\ \hline
{\em VII} & 3  & kod identyfikujący komórkę organizacyjną \\
		  &    & przedsiębiorstwa podmiotu leczniczego w  \\
		  &    & strukturze organizacyjnej tego podmiotu. \\ \hline
{\em VIII}& 4  & kod charakteryzujący specjalność komórki \\		
		  &    & organizacyjnej przedsiębiorstwa podmiotu \\
		  &    & leczniczego. \\ \hline
{\em IX}  & max. 9 & Kod funkcji ochrony zdrowia. \\ \hline
{\em X}   & 2  & Kod charakteryzujący dziedziny medycyny, \\
		  &    & których są udzielane świadczenia zdrowotne. \\ \hline
\end{tabular}
\end{minipage}
\end{table}

Screen z tego systemu co pokazuje info o podmiotach.

\section[Struktura przychodni medycznych][Struktura przychodni medycznych]{Struktura przychodni medycznych}


\section[Tytuł w paginie][Tytuł w spisie treści]{Pojęcia związane z prowadzeniem przychodni medycznej}
\subsection{Narodowy Fundusz Zdrowia}
\subsection{Personel}
\label{subsec:personel}
Personel w przychodni medyczne składa się z personelu medycznego oraz niemedycznego. Przykładowymi członkami pierwszej grupy są lekarze, lekarze dentyści, pielęgniarki, higienistki i asystentki. Do personelu niemedycznego należą pracownicy administracyjni, tacy jak dział księgowy, kadra zarządzająca, pracownicy rejestracji, ale też personel porządkowy. 
Pracownicy, którzy dopuszczeni są do kontaktu z pacjentem, muszą posiadać numer Prawa Wykonywania Zawodu (PWZ). Numer PWZ wydawany jest przez Naczelną Izbę Lekarską. Jest to 7 cyfrowa liczba, której pierwsza cyfra pełni funkcję kontrolną. W przypadku, gdy osoba posiada podwójny tytuł zawodowy np. lekarz i lekarz dentysta, posiada ona dwa numery PWZ. Pomimo, że taka sytuacja mogłaby się wydawać rzadkością, jest ona częstym przypadkiem wśród lekarzy starszych stażem. Wynika to z faktu, że przez reformą systemu edukacji lekarskiej, można było z łatwością uzyskać zarówno tytuł lekarza i lekarza dentysty.

Lekarze czynnie wykonujący zawód muszą znajdować się w Centralnym Rejestrze Lekarzy prowadzonym przez Naczelną Izbę Lekarską. Rejestr ten jest publicznie dostępny. Znajdują się w nim informacje takie jak imiona i nazwisko, numer PWZ, tytuł zawodowy i stan (aktualny bądź nie). Ponadto dostępne są dane na temat uzyskanych uprawnień i specjalizacji.
Po zmianach ustawowych w ostatnim czasie, do identyfikacji lekarza wystarcza jego imię, nazwisko oraz numer prawa wykonywania zawodu.

Numer prawa wykonywania zawodu muszą posiadać również pielęgniarki.

\subsection{Pacjent}
W myśl ustawy pacjentami są świadczeniobiorcami usług leczniczych. Światowa Organizacja Zdrowia uniezależnia pojęcie pacjent od faktycznego stanu zdrowia osoby. Inną definicją pacjenta jest jego stosunek do lekarza jako klient. W ostatnim czasie w Polsce obserwujemy proces prywatyzacji służby zdrowia. Ma to wprowadzić do tego sektora gospodarki mechanizmy wolnego rynku takie jak konkurencja. Taki proces powinien doprowadzić do zwiększenia efektywności rynku i zminimalizować nierentowność obecnie działających placówek.  

\subsection{Choroba - Międzynarodowa Klasyfikacja Chorób ICD-10}
\label{subsec:choroba}
Choroba w ogólnym znaczeniu jest to stan organizmu, który odbiega od normalnego. Z medycznego punktu widzenia jest to opis stanu pacjenta powiązany z konkretnymi objawami. Jej powodem mogą być zarówno zewnętrzne jak i wewnętrzne czynniki. Choroby mogą mieć przebieg ostry, podostry i przewlekły.

Światowa Organizacja Zdrowia (ang. World Health Organization - WHO) prowadzi Międzynarodową Klasyfikację Chorób i Powiązanych Problemów (ang. Internation Statistical Classification of Diseases and Related Health Problems). Klasyfikacja ta przyporządkowuje identyfikujące kody do chorób, symptomów czy uszkodzeń ciała. Zawiera ona miejsce na 14,400 różnych kodów i może być dalej rozszerzana. WHO udostępnia online szczegółowe informacje na temat swoje systemu. Przykładem jest internetowa przeglądarka kodów. Klasyfikacja jest bardzo często rozszerzana przez państwowe organizacje zajmujące się porządkowaniem chorób i procedur medycznych. Jej najnowszą rewizją jest dziesiąta (stąd ICD-10). Jest to standard uznawany na całym świecie.

W Polsce wszystkie podmioty lecznicze, które rozliczają się z NFZ, zobowiązane są do stosowania tej klasyfikacji. 

\subsection{Rozpoznanie}
\label{subsec:rozpoznanie}
Rozpoznanie jest identyfikacja prawdopodobnej choroby lub schorzenia na podstawie opinii lekarza. Jest wynikiem podejmowania testów klasyfikacyjnych. W celu dokonania diagnozy lekarz zbiera wywiad od pacjenta. Pozwala to ustalić potencjalne przyczyn schorzenia i zawęża obszar poszukiwań. Następnie diagnosta wykonuje badania. Możliwe jest również skierowanie pacjenta na wykonania badań w inne miejsce. Aby, tego dokonać konieczne jest stwierdzenie rozpoznania wstępnego. Lekarz stwierdza rozpoznanie główne (zasadnicze) występujące w klasyfikacji ICD-10. Rozpoznanie to jest wybierane na podstawie kosztów z nim związanych lub wagi dysfunkcji. Po zakończonym leczeniu rozpoznanie zasadnicze musi być określone. Dodatkowo mogą występować rozpoznania współistniejące, również określone kodem ICD-10. Są to schorzenia, które nie są głównym powodem leczenia, lecz występują w trakcie lub przed procesem hospitalizacji i mają wpływ na jej przebieg.

Każde rozpoznanie musi posiadać datę oraz opis wykonanych procedur po jakich stwierdzono daną chorobę.
\subsection{Skierowanie}
\label{subsec:skierowanie}
Skierowanie jest dokumentem wystawianym przez osoby kierujące na badanie lub konsultacje. Do skierowania dołącza się dokumentację medyczną niezbędną do wykonania badania lub konsultacji. Płatnikiem świadczeń udzielonych w ramach skierowania jest podmiot kierujący pacjenta. Wyniki badania lub konsultacji przekazywane są podmiotowi, który wystawił skierowania, przez podmiot je przeprowadzający. Zawartość skierowania określona jest przez Rozporządzenie Ministra \cite{mz:rzdm:online}. Rozporządzenie to wymienia następujące składniki:
\begin{itemize}
\item oznaczenie podmiotu wystawiającego skierowanie
\item oznaczenie pacjenta
\item oznaczenie rodzaju podmiotu, do którego kieruje się pacjenta na badanie lub konsultację
\item inne informacje lub dane, w szczególności rozpoznanie ustalone przez lekarza kierującego lub wyniki badań diagnostycznych, w zakresie niezbędnym do przeprowadzenia badania lub konsultacji
\item datę wystawienia skierowania
\item oznaczenie osoby kierującej na badanie lub konsultację
\end{itemize}

\subsection{Procedura - Międzynarodowa Klasyfikacja Procedur Medycznych ICD-9}
\label{subsec:procedura}
Procedura jest to usługa wykonana przez świadczeniodawce usług medycznych. Procedurą jest zarówno wykonanie operacji chirurgicznej jak i udzielenie porady lekarskiej.

Narodowy Fundusz Zdrowia udostępnia klasyfikację procedur, którą stosuje w celu rozliczania świadczeniodawców. Klasyfikacja ta unifikuje wszystkie procedury refundowane przez NFZ w jednolity system. Jest on również przydatna przy identyfikowaniu Jednolitych Grup Pacjentów (JGP). W JGP procedurom nadawane są rangi w zależności od znaczenia danej procedury dla procesu hospitalizacji.  Najwyższą rangę (>2) posiadają procedury podstawowe, czyli takie o ważnym charakterze zabiegowym, które ukierunkowują w stronę konkretnej grupy. Następnie występują procedury do wykonania w czasie leczenia o randze 2. Trzecią grupą są procedury dodatkowe, spełniające dodatkowe wymagania danych grup. Ostatnią grupą są procedury nieistotne, które nie wpływają na przydzieloną grupę.
Przykład fragmentu klasyfikacji procedur ICD-9-CM został zaprezentowany na rysunku \ref{pkpidc9}.

\begin{figure}[htb]
    \begin{center}
	\includegraphics[scale=.8]{img/fragment_klasyfikacji_icd9cm.pdf}
	\caption{Fragment klasyfikacji procedur ICD-9-CM}
	\label{pkpidc9}
    \end{center}
\end{figure}

NFZ nadaje wszystkim świadczeniom podlegającym refundacji określoną wartość kosztochłonności wyrażoną w punktach. Przykładem są "Cholecystektomia, usunięcie pęcherzyka żółciowego" o wartości 71 punktów lub "Ostre zapalenie trzustki o ciężkim przebiegu" warte 242 punkty. W momencie przetargu o kontrakt z NFZ świadczeniodawcy przedstawiają takie dane jak przygotowanie danej komórki do wykonywania świadczeń, dostępności wykonywanych świadczeń danym regionie oraz proponowaną cenę za punkt kosztochłonności. Pod koniec przetargu ustalana jest ostateczna cena i jest ona podstawą do refundacji wykonanych procedur. Kontrakt jest podpisywany z rozliczeniem rocznym. 
Podstawą do uzyskania zwrotu pieniędzy za świadczenia są listy refundowanych procedur dla danych rozpoznań. NFZ ściśle określa jakie czynności podlegają finansowaniu dla danego rozpoznania charakteryzowanego kodem ICD-10. 

\subsection{Wizyta}
\label{subsec:wizyta}
Wizyty, w przypadku przychodni lekarskich lub podmiotów leczniczych wykonujących ambulatoryjne świadczenia medyczne, są wykonywane w czasie krótszym niż 24 godziny. Zdecydowana ich większość to wizyty planowane. Praktycznie nie występują przyjęcia w trybie nagłym lub pilny, jak to ma miejsce w szpitalnej izbie przyjęć. Do wizyty przypisany jest pacjent oraz osoba personelu medycznego odpowiedzialna za wykonywane świadczenia medyczne. W przypadku, gdy osób personelu jest więcej niż jedna osoba, dokonuje się odpowiedniego wpisu do opisu przebiegu wizyty.
Rezerwacji wizyty dokonuje pracownik rejestracji. Notuje on przewidywaną procedurę jaka będzie udzielana na wizycie oraz informuje o tym personel odpowiedzialny. W trakcie wizyty lekarz wykonuje wywiad z pacjentem, a następnie przeprowadza procedury medyczne. Przed świadczeniami zdrowotnymi, które mogą niepożądanie spowodować uszczerbek na zdrowiu pacjenta, konieczne jest podpisanie przez niego odpowiednich klauzul. Dokumenty takie informują pacjenta o istniejących zagrożeniach związanych ze świadczeniami medycznymi. Ponadto, wykonywane są zdjęcia pacjenta zarówno przed jak i po zabiegu. Pod koniec wizyty przedstawiane są pacjentowi zalecenia, do których powinien się stosować. 

Ze względu na duże zainteresowania wizytami refundowanymi przez NFZ, konieczne jest tworzenie tzw. list oczekujących. Listy podzielona są na "przypadki pline" i "przypadki stabilne". Informacje zawarte w nich określają średni czas oczekiwania w dniach, liczbę osób oczekujących oraz liczba osób skreślonych z powodu wykonania świadczenia. Listy oczekujących są udostępniane dla NFZ, który publicznie publikuje takie informacje dla każdej z komórek organizacyjnych (np. poradnia specjalistycznej).

\subsection{Materiał}
\label{subsec:material}
W trakcie wykonywania niektórych procedur konieczne jest zużycie pewnych materiałów medycznych. Przykładem mogą być pościel dla pacjenta, zużycie materiałów opatrunkowych czy substancje specjalistyczne konieczne do przeprowadzenia operacji. Ze względu podmiotu lekarskiego istotne jest monitorowanie zużytych materiałów. Pozwala to zapanować nad kosztami oraz zmniejsza ryzyko podkradania materiałów przez pracowników.

\subsection{Płatnicy - Narodowy Fundusz Zdrowia}
\label{subsec:platnik}
W Polsce największym płatnikiem jest Narodowy Fundusz Zdrowia (NFZ). Pełni on funkcje ubezpieczyciela zdrowotnego. Jest to instytucja państwowa, która opłaca świadczenia zdrowotne z składek ubezpieczania zdrowotnego. Składki te są obowiązkowo opłacane przez każdego z obywateli Rzeczypospolitej Polskiej. W przypadku, gdy obywatel Polski nie jest w stanie samodzielnie opłacić składki, jest ona finansowana ze środków publicznych. Wynika to z zapisu w konstytucji, który określa, że każdy obywatel musi posiadać możliwość dostępu do darmowej opieki zdrowotnej. Niestety prowadzi to do bardzo długich kolejek oczekiwania na udzielenia nawet podstawowych świadczeń. Osoby, którym stan zdrowia niejednokrotnie nie pozwala tak długie oczekiwanie i tak zmuszone są z korzystania z prywatnej opieki zdrowotnej. Wyjątkiem są przypadki nagłe jak zagrożenie życia, przy których placówki medyczne mają obowiązek przyjąć pacjenta w trybie natychmiastowym. 
Ze względu na uprzywilejowaną (monopolistyczną) pozycję Narodowego Funduszu Zdrowia, na rynku płatników opłacanych ze składek publicznych, jego działanie jest nieefektywne. Środki są rozdysponowywane z interwałem rocznym, często w nieadekwatnych proporcjach dla świadczeniodawców. Proponowanym rozwiązaniem problemu jest wprowadzenie mechanizmu gry rynkowej wśród płatników. Problemem w tym rozwiązaniu jest wypłacalność ubezpieczycieli w sytuacjach osób narażonych na ryzyko "katastroficzne" (rzadka choroba).
Pomimo, większościowego udziału NFZ w finansowaniu świadczeń zdrowotnych, rolę płatników pełnią również osoby prywatne i płatnicy zewnętrzni.
Przykładami płatników zewnętrznych są firmy, które pokrywają całościowe lub częściowe koszty leczenia pracownika, lub ubezpieczyciele.

\subsection{Jednorodne Grupy Pacjentów}
Jednorodne Grupy Pacjentów (JGP) to systemem organizacji finansowania świadczeń zdrowotnych przez płatnika (NFZ). Jest on wdrażany w Polskim systemie ochrony zdrowia. W roku 2009 został wprowadzony w leczeniu szpitalnym.  Głównym czynnikiem, który skłonił do wprowadzenia tego systemy były wady jego poprzednika. W poprzednim systemie problemami były liczne nadużycia oraz brak powiązania klasyfikacji chorób i procedur z międzynarodowymi standardami (ICD-9, ICD-10).
System ten opiera się o podstawy teoretyczne opracowane przez Roberta B Fetter'a na Uniwersytecie w Yale. Jego głównym zadaniem jest przyporządkowywanie przypadków medycznych do określonych grup. Pobudką ku temu jest chęć wyselekcjonowania produktów jakie oferują świadczeniodawcy.
Grupowanie odbywa się na podstawie rozpoznania (ICD-10), wykonanych procedur, wieku i płci pacjenta oraz napotkanych komplikacji. Po dokonaniu takie podziału płatnik jest w stanie porównać jakie są koszty tych samych "produktów" w różnych placówkach medycznych.
NFZ prowadzi szkolenia mające na celu zaznajomienie świadczeniodawców z planowanym wdrożeniem JGP \cite{nfz:jgp:online}.

\subsection{Systemy zarządzania usługami medycznymi: eWUŚ i ZIP}
W 2013 Ministerstwo Zdrowia wdrożyło system Elektronicznej Weryfikacji Uprawnień Świadczeniobiorców (eWUŚ). Do systemu dostęp mają jedynie świadczeniodawcy. EWUŚ ma celu zapobieganie udzielaniu świadczeń zdrowotnych finansowanych ze środków publicznych osobom, które nie opłacają obowiązkowej składki zdrowotnej. 

Rozwiązaniem dla pacjentów jest Zintegrowany Informator Pacjenta (ZIP), który umożliwia im śledzenie historii swojego leczenia i jego finansowania począwszy od roku 2008. Ponadto, serwis ten zawiera informacje o przepisanych lekach oraz prawach do świadczeń zdrowotnych. ZIP wymaga rejestracji użytkowania. W momencie udostępnienia systemu w roku 2013 często okazywało się, że pacjenci mieli przypisane liczne procedury, których im w rzeczywistości nigdy nie udzielono. Było to powodem wyłudzania przez świadczeniodawców od NFZ refundacji za fikcyjnie udzielone świadczenia.

  

\section[Tytuł w paginie][Tytuł w spisie treści]{Dokumentacja medyczna}
\subsection{Podział dokumentacji medycznej}
Podział dokumentacji medycznej określony jest poprzez Rozporządzenie Ministra Zdrowia z dnia 21 grudnia 2010 r. w sprawie rodzajów i zakresu dokumentacji medycznej oraz sposobu jej przetwarzania \cite{mz:rzdm:online}
Rozporządzenie to wyróżnia dokumentację indywidualną oraz zbiorcza. Dokumentacja indywidualna dotyczy poszczególnych pacjentów korzystających ze świadczeń zdrowotnych, zbiorcza odnosi się do ogółu pacjentów lub określonych grup pacjentów.  

Dokumentacja indywidualna dzieli się na dokumentację wewnętrzną oraz zewnętrzną. Dokumentacja wewnętrzna przeznaczona jest na potrzeby podmiotu udzielającego świadczeń zdrowotnych. W jej skład wchodzą:
\begin{itemize}
\item historia zdrowia i choroby
\item historia choroby
\item karta noworodka
\item karta indywidualnej opieki pielęgniarskiej
\item karta indywidualnej opieki prowadzonej przez położną
\item karta wizyty patronażowej
\item karta wywiadu środowiskowo–rodzinnego
\end{itemize}
Dokumentacja zewnętrzna przeznaczona jest na potrzeby pacjenta korzystającego ze świadczeń zdrowotnych udzielanych przez podmiot. Jest ona złożona z:
\begin{itemize}
\item skierowanie do szpitala lub innego podmiotu
\item skierowanie na badanie diagnostyczne lub konsultację
\item zaświadczenie, orzeczenie, opinia lekarska
\item karta przebiegu ciąży
\item karta informacyjna z leczenia szpitalnego
\end{itemize}

\subsection{Zawartość dokumentacji}

Wymagania co do zawartości dokumentacji medycznej są różne w zależności od podmiotu leczniczego. Ze względu na wymaganą dokumentację medyczną można wyróżnić następujące podmioty lecznicze: szpital; zakład przeznaczony dla osób, których stan zdrowia wymaga udzielania całodobowych lub całodziennych świadczeń zdrowotnych; zakład opieki zdrowotnej, w którym czas pobytu pacjenta niezbędny do udzielenia świadczenia zdrowotnego nie przekracza 24 godzin; zakład opieki zdrowotnej udzielający świadczeń zdrowotnych w warunkach ambulatoryjnych; podmiot sprawujący opiekę nad kobietą ciężarną; żłobek; pracownie diagnostyczne; dysponent zespołów ratownictwa medycznego; pracownia protetyki stomatologicznej i ortodoncji;  zakład rehabilitacji leczniczej; lekarz lub pielęgniarka udzielająca świadczeń zdrowotnych w ramach indywidualnej praktyki, indywidualnej specjalistycznej praktyki albo grupowej praktyki; lekarz podstawowej opieki zdrowotnej, pielęgniarka lub higienistka szkolna udzielający świadczeń zdrowotnych uczniom.

Ta praca dotyczy zakładów opieki zdrowotnej udzielających świadczeń zdrowotnych w warunkach ambulatoryjnych. Dla zakładów tego typu wymagana dokumentacja indywidualna składa się z historii zdrowia i choroby, skierowań do szpitala lub innego podmiotu, skierowań na badanie diagnostyczne lub konsultację, zaświadczeń, orzeczeń, opinii lekarskich, kart przebiegu ciąży.  Z kolei dokumentacja zbiorcza przyjmuje formę księgi przyjęć, kartoteki środowisk epidemiologicznych, księgi pracowni diagnostycznej, księgi zabiegów prowadzonej odrębnie dla każdego gabinetu zabiegowego, księgi porad ambulatoryjnych dla nocnej i świątecznej pomocy lekarskiej i pielęgniarskiej. 

 
Rozporządzenie narzuca wymagania dotyczące wpisów do dokumentacji medycznej. Mianowicie, każdy wpis musi być zawierać oznaczenie osoby dokonującej go. Wpisu w dokumentacji medycznej dokonuje się natychmiast po udzieleniu świadczenia zdrowotnego. Musi być czytelny i ułożony w porządku chronologicznym. Z dokumentacji nie można usuwać wpisów. W przypadku błędów, zamieszcza się adnotację o przyczynie błędu oraz datę i oznaczenie osoby dokonującej adnotacji. Strony muszą być numerowane i oznaczone imieniem i nazwiskiem pacjenta.

\subsection[Wymagania związane z przechowywaniem dokumentacji medycznej][Wymagania związane z przechowywaniem dokumentacji medycznej]{Wymagania związane z przechowywaniem dokumentacji medycznej}

Rozporządzenie nakłada również wymogi odnośnie przechowywania dokumentacji medycznej. Dokumentacja wewnętrzna przechowywana jest przez podmiot, który ją sporządził, podczas gdy dokumentacja zewnętrzna przechowywana jest przez podmiot, który zrealizował zlecone świadczenie zdrowotne.

Dla dokumentacji prowadzonej w postaci elektronicznej sprecyzowane są szczegółowe wymagania. W szczególności konieczna jest możliwość eksportu całości danych w formacie XML, tak aby możliwe było odtworzenie dokumentacji w innym systemie teleinformatycznym. W przypadku takiego przeniesienia zapisywana jest data przeniesienia oraz informację o tym z jakiego systemu dokonany został transfer danych. Forma dokumentacji powinna być zgodna z tą określoną przez normy dotyczące gromadzenia i wymiany informacji w ochronie zdrowia. System teleinformatyczny elektronicznej dokumentacji medycznej musi posiadać możliwość drukowania dokumentacji w formatach określonych w rozporządzeniu. Wszelkie dokumenty utworzone w postaci innej niż elektroniczna (np. zdjęcia radiologiczne) mają być odwzorowane do postaci cyfrowej.


Ze względu na to, że dokumentacja medyczne przechowuje dane osobowe konieczne jest przestrzeganie odpowiednich wymogów prawnych.
Według Biura Rzecznika Praw Obywatelskich \cite{mz:do:online} dane osobowe to wszelkie informacje dotyczące zidentyfikowanej lub możliwej do zidentyfikowania osoby fizycznej. Osobą możliwą do zidentyfikowania jest osoba, której tożsamość można określić bezpośrednio lub pośrednio, w szczególności przez powołanie się na numer identyfikacyjny albo jeden lub kilka specyficznych czynników określających jej cechy fizyczne, fizjologiczne, umysłowe, ekonomiczne, kulturowe lub społeczne. Każdy ma prawo do ochrony dotyczących go danych osobowych. Przetwarzanie danych osobowych to wszystkie operacje wykonywane na danych osobowych, takie jak zbieranie, utrwalanie, przechowywanie, opracowywanie, zmienianie, udostępnianie i usuwanie, a zwłaszcza te, które wykonuje się w systemach informatycznych.

Powyższa definicja wskazuje na to, że w systemie dedykowanym dla podmiotów leczniczych mam do czynienia z przetwarzaniem danych osobowych. W związku z tym konieczne jest przestrzeganie wymagań określonych w rozporządzeniu ministra spraw wewnętrznych i administracji z dnia 29 kwietnia 2004 r. w sprawie dokumentacji przetwarzania danych osobowych oraz warunków technicznych i organizacyjnych, jakim powinny odpowiadać urządzenia i systemy informatyczne służące do przetwarzania danych osobowych \cite{mz:odo:online}. 

Rozporządzenie to określa sposób prowadzenia i zakres dokumentacji opisującej sposób przetwarzania danych osobowych oraz środki techniczne i organizacyjne zapewniające ochronę przetwarzanych danych osobowych
odpowiednią do zagrożeń oraz kategorii danych objętych ochroną.
Nakłada ono wymaganie prowadzenie dokumentacji w postaci poniższych dokumentów.
Polityki bezpieczeństwa, która zawiera: 
\begin{itemize}
\item wykaz budynków, pomieszczeń lub części pomieszczeń, tworzących obszar, w którym przetwarzane są dane osobowe;
\item wykaz zbiorów danych osobowych wraz ze wskazaniem programów zastosowanych do przetwarzania tych danych;
\item opis struktury zbiorów danych wskazujący zawartość poszczególnych pól informacyjnych i powiązania między nimi;
\item sposób przepływu danych pomiędzy poszczególnymi systemami;
\item określenie środków technicznych i organizacyjnych niezbędnych dla zapewnienia poufności, integralności i rozliczności przetwarzanych danych.
\end{itemize}
Instrukcji zarządzania systemem informatycznych składającej się z:
\begin{itemize}
\item procedury nadawania uprawnień ń do przetwarzania danych i rejestrowania tych uprawnień w systemie informatycznym oraz wskazanie osoby odpowiedzialnej za te czynności;
\item stosowane metody i środki uwierzytelnienia oraz procedury związane z ich zarządzaniem i użytkowaniem;
\item procedury rozpoczęcia, zawieszenia i zakończenia pracy przeznaczone dla użytkowników systemu;
\item procedury tworzenia kopii zapasowych zbiorów danych oraz programów i narzędzi programowych służących do ich przetwarzania;
\item sposób, miejsce i okres przechowywania: a) elektronicznych nośników informacji zawierających dane osobowe, b) kopii zapasowych
\item sposób zabezpieczenia systemu informatycznego przed działalnością oprogramowania;
\item procedury wykonywania przeglądów i konserwacji systemów oraz nośników informacji służących do przetwarzania danych.
\end{itemize}

Rozporządzenie stanowi, że gdy przynajmniej jedno urządzenie systemu informatycznego, służące do przetwarzania danych osobowych, połączone jest z siecią publiczną, stosuje się poziom bezpieczeństwa wysoki (najwyższy). Taka sytuacja występuje w projektowanym webowym systemie. 

Środki bezpieczeństwa przewidziane dla wysokiego poziomu ryzyka obejmują:
\begin{itemize}
\item Osoby nieuprawnione do przebywania w obszarze, gdzie następuje przetwarzanie danych osobowych, muszą posiadać na to zgodę administratora danych albo być w obecności osoby uprawnionej.
\item Konieczne są mechanizmy kontroli dostępu, w których każdy użytkownik posiada odrębny identyfikator i wymagane jest uwierzytelnianie zabezpieczone środkami kryptograficznymi.
\item System musi być zabezpieczony przed nieuprawnionym dostępem, bądź utratą danych wynikających z problemów z zasilaniem.
\item Nie można użyć drugi raz przydzielonego już kiedyś identyfikatora
\item Hasło użytkownika, składa się z co najmniej 8 znaków, zawiera małe i wielkie litery oraz cyfry lub znaki specjalne. Jego zmiana ma następować nie rzadziej niż co 30 dni.
\item Przepływ informacji pomiędzy systemem informatycznym a siecią publiczną jest kontrolowany
\end{itemize}


\subsection{Kompatybilność z innymi systemami}
\section[Przegląd istniejących rozwiązań][Przegląd istniejących rozwiązań]{2 Przegląd istniejących rozwiązań}
Podmioty lecznicze udzielają często bardzo różnorodnych świadczeń zdrowotnych. Posiadają one wyspecjalizowane potrzeby. Przykładem może być porównanie gabinetu stomatologicznego i szpitala. Gabinet stomatologiczny wymaga prowadzenia dokumentacji z uwzględnieniem konkretnych zębów. Z tego względu w takich systemach popularnych rozwiązaniem jest stosowanie graficznego obrazu jamy ustnej i \emph{wyklikiwanie} na nim rozpoznanych chorób oraz udzielonych procedur. W przypadku szpitala konieczne jest uwzględnienie procesów, w których pacjenci hospitalizowani są przez długi okres czasu lub procesów na wypadek śmierci pacjenta. Rozwiązaniem, które stara się rozwiązać problem takiej różnorodności, jest tworzenie specjalistycznych modułów, które odpowiadają za określoną funkcjonalność. Przekładami są moduł stomatologiczny, księgowość, moduł menadżerski czy pracowni diagnostycznej. Aplikacje wdrażane są dla indywidualnych potrzeb klientów poprzez wykorzystanie stosownych modułów. Powoduje to większą elastyczność rozwiązań jednak zwiększa nakłady związane z wdrożeniem systemu. 

Aktualnie na rynku aplikacji do wspierania pracy podmiotów leczniczych występują jedynie duże rozwiązania komercyjne realizowane przez Polskie firmy i na Polski rynek. Wyłączność Polskich dostawców oprogramowania wynika z licznych wymogów Polskiego ustawodawcy, które utrudniają stworzenie aplikacji spełniających wymogi zarówno Polskie jak i innego kraju. Aplikacje, które są obecnie dostępne mają w głównej mierze charakter desktopowy. Specyfika takiego rozwiązania wynika z pobierania opłat za oprogramowanie w zależności od ilości stanowisk użytkowych. W przypadku aplikacji działających w przeglądarce internetowej wystąpiłby problem z kontrolowaniem ilości takich stanowisk.  
\subsection{KamSoft-SOMED}
Pierwszym z omawianych rozwiązań jest program SOMED firmy KamSoft. Rozwiązanie to posiada obszerną funkcjonalność. Stworzonych jest wiele modułów, które mogą zostać dostosowane w potrzeb użytkownika. Wadą jest jednak wysoka cena oraz skomplikowany interfejs użytkownika. Na rysunku \ref{kamsoft_menu} przedstawione są moduły dostępne po uruchomieniu wersji demonstracyjnej programu.

\begin{figure}[htb]
    \begin{center}
	\includegraphics[scale=.6]{img/przeglad/kamsoft_menu.jpeg}
	\caption{Ekran główny i przegląd dostępnych modułów w Kamsoft-SOMED}
	\label{kamsoft_menu}
    \end{center}
\end{figure}

\subsection{Assecco mMedica}
Rozwiązanie to jest już dobrze zakorzenione na Polskim rynku. Jest najbardziej popularne wśród podmiotów leczniczych. Jak ciekawostkę można dodać, że tego narzędzia uczeni są studenci kierunku lekarskiego i lekarsko-dentystycznego w ramach zajęć Informatyka na Warszawskim Uniwersytecie Medycznym. MMedica posiada przyjazdy i intuicyjny interfejs użytkownika. Posiada rozbudowany moduł do rozliczeń świadczeń z NFZ. 
Na rysuknu \ref{assecco_mmedica} widoczny jest ekran główny aplikacji mMedica firmy Assecco. 
\begin{figure}[htb]
    \begin{center}
	\includegraphics[scale=.6]{img/przeglad/assecco_mmedica.jpeg}
	\caption{Ekran główny w Assecco mMedica}
	\label{assecco_mmedica}
    \end{center}
\end{figure}
Na rysunku \ref{assecco_mmedica_widok_aplikacji} przedstawione ekran edycji wizyty dostępny dla personelu medycznego.
\begin{figure}[htb]
    \begin{center}
	\includegraphics[scale=.23]{img/przeglad/assecco_mmedica_widok_aplikacji.jpeg}
	\caption{Ekran edytora wizyty w Assecco mMedica}
	\label{assecco_mmedica_widok_aplikacji}
    \end{center}
\end{figure}

\subsection{EuroSoft mPrzychodnia}
Program jest nowym rozwiązaniem. Tworzony jest z myślą o zapewnieniu użytkownikom pełnej zgodności z wymaganiami dotyczącymi ustawy o elektronicznej dokumentacji medycznej. Ze względu na ciągłe jeszcze dopracowywanie szczegółów implementacyjnych nie posiada ono kilku kluczowych funkcjonalności. Przykładem jest brak odnotowywania użytkownika, który dokonał modyfikacji danych medycznych, co jest wymagane przez regulatora. Wersja demo mPrzychodni dostępna jest jedynie po odbyciu szkolenia. Ponadto instalowana jest ona wyłącznie na jednym komputerze poprzez zdalne połączenie się pracownika tej firm do komputera potencjalnego klienta i zainstalowanie niego klucza pozwalającego na korzystanie z programu. 
Na rysunku \ref{eurosoft_widok_aplikacji} przedstawiono widok listy przyjęć pacjenta i edytor przyjęcia pacjenta w systemie mPrzychodnia firmy Eurosoft.
\begin{figure}[htb]
    \begin{center}
	\includegraphics[scale=.32]{img/przeglad/eurosoft_widok_aplikacji.jpg}
	\caption{Widok listy przyjęć i edytora przyjęć pacjenta w Eurosoft mPrzychodnia}
	\label{eurosoft_widok_aplikacji}
    \end{center}
\end{figure}


